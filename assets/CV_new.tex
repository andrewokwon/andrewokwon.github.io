\documentclass{article}
\usepackage{akwon}
\usepackage{hyperref}
\hypersetup{colorlinks=true,urlcolor=cyan}
\usepackage{palatino}
\urlstyle{same}
\setlength{\parindent}{0pt}
\setlength{\marginparpush}{0pt}
\setlength{\marginparwidth}{85pt}
\begin{document}

\LARGE{Andrew Kwon}\vspace{0.0in}\\

\normalsize
Department of Mathematics\\
University of Pennsylvania\\
David Rittenhouse Laboratory\\
Philadelphia, PA 19104\\

email: \href{mailto:akwon@upenn.edu}{akwon@upenn.edu}\\
website: \url{https://andrewokwon.github.io}\vspace{0.25in}\\
\reversemarginpar
\Large{Education}\vspace{0.2in}\\
\normalsize
PhD in Mathematics, University of Pennsylvania\marginpar[\raggedleft Expected 2025]{}\\
{\small{Advisor: Florian Pop}}\\

BSc in Mathematics, Carnegie Mellon University\marginpar[\raggedleft 2019]{}
\vspace{0.25in}

\Large{Awards and Honors}\vspace{0.2in}\\
\normalsize
\marginpar[\raggedleft 2019 - 2024]{}NSF Graduate Research Fellowship\\
\marginpar[\raggedleft 2018]{}Goldwater Scholar\\
\marginpar[\raggedleft 2017]{}William Lowell Putnam Competition Honorable Mention
\vspace{0.25in}

\Large{Publications \& Preprints}\vspace{0.2in}\\
\normalsize
A. Kwon, \textit{Convergent Decomposition Groups and the $\mathfrak{S}$-adic Shafarevich Conjecture}, in preparation.\\

A. Kwon, \textit{A Note on Diophantine Subsets of Large Fields,} preprint.\\

A. Kwon, \textit{A Note on Minimal Additive Complements,} Discrete Mathematics, 342 (7) (2019) 1912-1918.\\

M. Asada, R. Chen, E. Fourakis, Y. Kim, A. Kwon, J. D. Lichtman, B. Mackall, S. J. Miller, E. Winsor, K. Winsor, J. Yang, and K. Yang, \textit{Lower-Order Biases in Second Moments of Dirichlet Coefficients in Families of $L$-Functions,} Experimental Mathematics, 32 (3), 431-456.

\vspace{0.25in}
%\Large{Talks}\\
%\normalsize
%\begin{itemize}
%	\item Joint Mathematics Meetings, San Diego, CA 2018. 
%	\item Young Mathematicians Conference, The Ohio State University, Columbus, OH 2016.
%\end{itemize}

\newpage
\Large{Teaching}\vspace{0.2in}\\
\normalsize
\marginpar[\raggedleft{Summer 2023}]{}\textit{Instructor of Record}\\
	University of Pennsylvania\\
	\small{Designed lectures, problem sets, and assessments for Introduction to Calculus.}\normalsize\\
	
	\normalsize\textit{Teaching Assistant}\marginpar[\raggedleft{2020 - 2021}]{}\\
	\marginpar[\raggedleft{2024 - 2025}]{}University of Pennsylvania\\
	\small{Taught section/graded problem sets for Multivariable Calculus}.\\
	
	\normalsize\textit{Teaching Assistant}\marginpar[\raggedleft{2016 - 2019}]{}\\
	Carnegie Mellon University\\
	\small{Taught section/graded problem sets for: Concepts of Mathematics (3x), Linear Algebra.} \normalsize\\

\vspace{0.25in}
\Large{Employment}\vspace{0.2in}\\
\normalsize
\marginpar[\raggedleft{2018}]{}\textit{Research Intern}\\
	National Security Agency\\
	\small{Applied and extended language modeling and $n$-gram techniques to a high-priority classified project.}
	\normalsize

\vspace{0.25in}
\Large{Service and Activities}\vspace{0.2in}\\
\normalsize
Co-organizer\marginpar[\raggedleft{2024 - 2025}]{}\\
Penn-Temple Algebra Day\\

Academic Committee\marginpar[\raggedleft{2024}]{}\\
GTA Philadelphia Conference at Temple\\

Co-organizer\marginpar[\raggedleft{2022 - 2024}]{}\\
Graduate Algebraic Geometry Seminar, University of Pennsylvania\\
	
Instructor\marginpar[\raggedleft{2021}]{}\\
Princeton Prison Teaching Initative\\
	
Mentor\marginpar[\raggedleft{2021 - 2024}]{}\\
Penn Directed Reading Program\\
	
Teaching Assistant\marginpar[\raggedleft{2020, 2022}]{}\\
Penn Summer Math Academy\\
	
Referee\marginpar[\raggedleft{2018 - 2020}]{}\\
Journal of Number Theory, Publicationes Mathematicae Debrecen
	
%Co-founder, Vice President\marginpar[\raggedleft{2015 - 2018}]{}\\
%Carnegie Mellon Informatics and Mathematics Competition\\
\end{document}
