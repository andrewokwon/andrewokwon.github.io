\documentclass{article}
\usepackage{akwon}
\setlength{\parindent}{0pt}
\title{Homework 3 feedback}
\author{18/20}
\date{}
\begin{document}
\maketitle
\begin{enumerate}
	\item Good!
	\item Good!
	\item In the same way that we can calculate the derivative of $\operatorname{arctan}(x) = \frac{1}{1+x^{2}}$, we can calculate that the derivative of $\operatorname{arccot}(x) = \frac{-1}{1+x^{2}}$. If we let $y = \operatorname{arccot}(x)$ then $\cot y = x$, and since $1 + \cot^{2} y = \csc^{2} y$, we have $\csc^{2} y = \csc^{2}(\operatorname{arccot} x) = 1 + \cot^{2} (\operatorname{arccot} x) = 1 + x^{2}$.
	\item Good!
	\item Good!
	\item Good! 
	\item (1/2) We discussed part (c) in office hours, but here's another way to see why this case cannot happen. Notice that a concave up function (namely, a function where $f''(x) > 0$ for all $x$) will always lie above any tangent line. However, basically any tangent line will eventually go above the $x$-axis, and thus the function must eventually do this also. Thus, the condition $f''(x) > 0$ is incompatible with the condition $f(x) < 0$.
	\item (1/2) I don't recognize the method used to calculate the derivative. It should be 
		\[
			f'(x) = 1 - \left( \frac{4 (x+1) - (4x)(1)}{(x+1)^{2}} \right) = 1 - \frac{4}{(x+1)^{2}},
		\]
		and then solving for critical points yields $(x+1)^{2} = 4 \Leftrightarrow x = -3, 1$. However, only $x=1$ is in the domain, and then we can check that $f(1) = -1$, while $f(0) = f(3) = 0$, so $-1$ is the minimum value of the function on $[0,3]$.
	\item Good!
\end{enumerate}
\end{document}
