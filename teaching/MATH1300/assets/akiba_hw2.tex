\documentclass{article}
\usepackage{akwon}
\setlength{\parindent}{0pt}
\title{Homework 2 feedback}
\author{13/20}
\date{}
\begin{document}
\maketitle
\begin{enumerate}
	\item (1/2) For part (c), the derivative of $2^{x}$ is not $2x$. The derivative of $x^{2}$ is $2x$. The way we compute the derivative of exponential functions is via $e^{x}$ and the chain rule:
		\[
			(2^{x})' = ( (e^{\log(2)})^{x} )' = (e^{x \log 2})' = e^{x \log 2} \cdot \log 2.
		\]
	\item Good!	
	\item (1/2) The calculation of the limit wasn't really finished here\ldots Part of the purpose of this problem is to actually verify that the power rule works, since in-class we only verified the power rule works for $x^{1/2}$. The way we calculate the limit is via
		\begin{align*}
		\frac{\sqrt[3]{x+h} - \sqrt[3]{x}}{h} \cdot \frac{\sqrt[3]{x+h}^{2} + \sqrt[3]{x+h} \sqrt[3]{x} + \sqrt[3]{x}^{2}}{\sqrt[3]{x+h}^{2} + \sqrt[3]{x+h} \sqrt[3]{x} + \sqrt[3]{x}^{2}} &= \frac{\sqrt[3]{x+h}^{3} - \sqrt[3]{x}^{3}}{h \cdot (\sqrt[3]{x+h}^{2} + \sqrt[3]{x+h} \sqrt[3]{x} + \sqrt[3]{x}^{2})}\\
		&= \frac{1}{\sqrt[3]{x+h}^{2} + \sqrt[3]{x+h} \sqrt[3]{x} + \sqrt[3]{x}^{2}},
		\end{align*}
		and taking $h \to 0$ we find that the limit is equal to $\frac{1}{3 \sqrt[3]{x^{2}}} = \frac{1}{3} x^{-2/3}$.
	\item For part (c), it isn't actually possible to evaluate $\frac{x \cos(x) - \sin(x)}{x^{2}}$ at $x = 0$, since then we get $\frac{0}{0}$. There is a solution to this problem using the squeeze theorem, and although it only uses material we've covered so far in class, this problem was harder than I thought. (I had a mistake in the original solution that made it seem easier.) I can discuss the solution in detail, upon request.
	\item Good!
	\item (1/2) When implicitly differentiating the equation, we should get
		\[
			\cos(\pi xy) \cdot (\pi y + \pi x y') = \pi(1 + y'),
		\]
		where we used the chain rule for $\sin(\pi xy)$, and then we have to use product rule to compute $(\pi xy)' = \pi y + \pi x y'$. With this formula we should get the answer $y' = 0$. (If you graph this equation in Desmos or something then you'll also see this is the case.)
	\item (0/2) I don't understand equations such as 
		\[
			\operatorname{arcsin}(0) = 1/\sqrt{1} (1 - u^{2}).
		\]
		Apparently the chain rule is in mind here, but we have to fully compute the derivatives first, and then evaluate them at specific values. For example, the derivative of $\operatorname{arccos}(2 \sqrt{x})$ is 
		\[
			\frac{-1}{\sqrt{1 - (2 \sqrt{x})^{2}}} \cdot (2 \sqrt{x})' = \frac{-1}{\sqrt{1 - 4x}} \cdot (2 \cdot \frac{1}{2} x^{-1/2}) = \frac{-1}{\sqrt{1 - 4x}\sqrt{x}}.
		\]
		For part (b), the derivative of $\operatorname{arctan}(x)$ is $\frac{1}{1+x^{2}}$, and so the derivative of $\operatorname{arctan}(x^{2})$ is
		\[
			\frac{1}{1 + (x^{2})^{2}} \cdot (x^{2})' = \frac{2x}{1 + x^{4}}.
		\]
		For part (c), the derivative of $\operatorname{arcsin}(x)$ is $\frac{1}{\sqrt{1-x^{2}}}$, and the derivative of $\log(x) = \frac{1}{x}$, and so the derivative of $\operatorname{arcsin}(\log(x))$ is 
		\[
			\frac{1}{\sqrt{1 - (\log(x))^{2}}} \cdot (\log(x))' = \frac{1}{\sqrt{1 - (\log(x))^{2}}} \cdot \frac{1}{x}.
		\]
	\item (0/2) The way this problem should be done is by finding the general equation of a line tangent to the parabola $y = x^{2}$ at some point $(a, a^{2})$, and finding when this line passes through the point $(0, -1)$. The slope of the graph at $x = a$ is $2a$, and so the equation of the tangent line is
		\[
			(y - a^{2}) = (2a)(x-a) \Leftrightarrow y = 2ax + (a^{2} - 2a^{2}).
		\]
		In order for the $y$-intercept to be $-1$, we need $-a^{2} = -1$, or in other words $a^{2} = 1$. Thus, $a = \pm 1$, and the two tangent lines are given by
		\[
			y = 2x - 1, y = -2x -1.
		\]
\end{enumerate}
\end{document}
